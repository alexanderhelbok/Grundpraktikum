% !TeX root = 00_Vorlage.tex
% !TeX spellcheck = de_DE
\section{Theorie}
\label{sec:theorie}
Im nachfolgenden Kapitel wird auf die theoretischen Grundlagen des Versuches eingegangen.

\subsection{Trigonometrische Überlegungen}
Der Betrag eines Vektorfeldes in kartesischen Koordinaten lässt sich mit dem Satz vom Pythagoras als Wurzel der Quadratsummen anschreiben und schaut wie folgt aus
\begin{equation}\label{eqn:Babs}
	|\vec{B}| = \sqrt{B_x^2 + B_y^2 + B_z^2}
\end{equation}
, wobei \( B_i \) die kartesischen Komponenten des Vektorfeldes darstellen.

Betrachtet man den Winkel zwischen einer Ebene und einem Vektorfeld, lässt sich das über das Skalarprodukt zwischen dem Vektorfeld und dessen Projektion auf die Ebene bewerkstelligen. Dieser Ausdruck lässt sich vereinfachen, wenn man das Koordinatensystem so dreht, dass eine Komponente des Vektorfelds 0 wird. Setzt man zum Beispiel die y-Komponente auf 0, erhält man für den Winkel 
\begin{equation}\label{eqn:theta}
	\theta = \arctan(\frac{B_z}{B_x})
\end{equation}
.

\subsection{Einführung in die Elektrodynamik}
Alle in diesem Abschnitt beschriebenen Konzepte lassen sich im Kapitel 3 vom Demtröder nachlesen \cite{demtröder}.
Ein großer Teilbereich der Physik stellt die Elektrodynamik dar, welche sich mit der Studie von Ladungen in elektrischen und magnetischen Feldern beschäftigt. Von zentraler Bedeutung ist hierbei die Lorentzkraft 
\begin{equation}\label{eqn:FL}
	\vec{F}_{\text{L}} = q\left(\vec{E} + \vec{v} \times \vec{B}\right)
\end{equation}
, die eine bewegte Ladung \( q \) mit Geschwindigkeit \( \vec{v} \) in einem elektrischen Feld \( \vec{E} \), sowie magnetischen Feld \( \vec{B} \) verspürt. To note! ist hier, dass der Betrag von E-Feld immer parallel zu den Feldlinien verläuft, während das B-Feld eine Kraft erzeugt, welche normal auf die Feldlinien (und die Geschwindigkeit der Ladung) steht. Die tatsächliche Richtung der Lorentzkraft kann man sich dann mit der Rechten-Hand-Regel herleiten, wobei der Daumen \( \vec{v} \), der Zeigefingern \( \vec{B} \) und der Mittelfinger \( \vec{F}_{\text{L}} \) repräsentiert. 

Das Vorzeichen der Kraft wird (mitunter) durch die Ladung bestimmt, was in Stromkreisen zu Verwirrung führen kann. In diesem Bericht wird die technische Stromrichtung gewählt, sodass die Ladungsträger in einem Kabel positiv sind und der Strom somit vom Pluspol zum Minuspol fließt. 

Betrachtet man einen elektrischen Strom in einem Leiter und setzt man \( q\vec{v} = I\vec{l} \) man erhält für den zweiten Teil der Lorentzkraft
\begin{equation}\label{eqn:FL2}
	\vec{F}_{\text{L}} = I(\vec{l} \times \vec{B})
\end{equation}
. Hierbei ist \( I \) die Durchflussrate der Ladungen und \( \vec{l} \) der Längenvektor des Leiters. 

Mit dem Biot-Savart Gesetz lässt sich das von einem Strom erzeugte Magnetfeld über ein Kurvenintegral ausdrücken und sieht folgendermaßen aus
\begin{equation}\label{eqn:biot}
	\vec{B}(r) = \frac{\mu_0}{4\pi} \uint[C]{\frac{I}{|\vec{r}|^3}}{\vec{l} \times \vec{r}}
\end{equation}
, mit dem differentiellen Längenelement \( \dd{\vec{l}} \), dem Ortsvektor \( \vec{r} \) und der magnetischen Feldkonstante \( \mu_0 \). Wählt man für die Kurve \( C \) einen Kreis und schaut man sich nur den Absolutbetrag des Magnetfeldes an, vereinfacht sich das Integral zu 
\begin{equation}\label{eqn:last}
	|\vec{B}(r)| = \frac{\mu_0 I}{2\pi}\frac{1}{r} = k\tilde{r}
\end{equation}
wobei der konstante Vorfaktor durch \( k := \tfrac{\mu_0 I}{2\pi} \) ersetzt wurde. Da man aus Geraden leichter Werte, wie die Steigung ablesen kann, wurde \( \tilde{r} := 1/r \) eingeführt.

\subsection{Statistische Grundlagen}
Hier werden kurz die Methoden erwähnt, welche für die statistische Aufbereitung der Daten essentiell sind.

Das Arithmetische Mittel, auch Mittelwert oder Durchschnitt genannt, ist das wohl meist verwendete Werkzeug der Statistik. Es lässt sich sowohl  für „exakte“, als auch für fehlerbehaftete Daten definieren. Im ersten Fall spricht man von einem ungewichteten Mittelwert und man schreibt 
\begin{equation}\label{eqn:mean1}
	\bar{x} = \frac{1}{N} \sum_{i=1}^{N} x_i
\end{equation}
mit $\bar{x}$ als Mittelwert von \( N \) Daten \cite{error}.

Sind die Daten Fehlerbehaftet (mit Fehler \( \alpha_i \)) muss man diesen Berücksichtigen und erhält
\begin{equation}\label{eqn:mean2}
	\bar{x} = \left(\sum_{i=1}^{N} \frac{1}{\alpha_i^2}\right)^{-1} \sum_{i=1}^{N} \frac{x_i}{\alpha_i^2}
\end{equation}
Haben alle Daten aber den selben Fehler, kürzt sich dieser weg und man landet wieder bei \autoref{eqn:mean1}. \cite[S. 50]{error} Aus diesem Grund wird in diesem Versuch großteils der ungewichtete Mittelwert angewandt, obwohl alle Daten fehlerbehaftet sind. 

Die Standardabweichung ist ein direktes Maß für die Verteilung der Daten und gibt an, wie weit die Daten im Mittel vom Durchschnitt abweichen. Sie berechnet sich wie folgt
\begin{equation}\label{eqn:sigma}
	\sigma = \SQRT{\frac{1}{N-1} \sum_{i=1}^{N}(x_i - \bar{x})^2}
\end{equation}
Wir dividieren hier durch \( N-1 \), da der Mittelwert, der in der Berechnung der Standardabweichung herangezogen wird, die \( N \) Freiheitsgrade der  \( N \) Daten um einen reduziert \cite{error}.

Die Kombination der Unsicherheiten von fehlerbehafteten Messdaten erfolgt mittels der Gaußschen Fehlerpropagation. Diese stellt einen Zusammenhang zwischen dem Fehler der Größe \( Z(x_i) \), welche von \( x_i \) Variablen abhängt, und den partiellen Ableitungen nach \( x_i \) her. In allgemeiner Form sieht die Formel wie folgt aus
\begin{equation}\label{eqn:propagation}
	\alpha_Z = \SQRT{\sum_{i=1}^{N}\left( \pdv{Z}{x_i} \alpha_{x_i}\right)^2}
\end{equation}
mit \( \alpha_{x_i} \) als Fehler der einzelnen Größen, von denen \( Z \) abhängt \cite{error}. Oft wird für \( \alpha_{x_i} \) die Standardabweichung aus \autoref{eqn:sigma} verwendet, das muss aber nicht der Fall sein.
Die Propagation von Fehlern erfolgt in diesem Versuch automatisch und wird im Hintergrund gehalten.

%Eine kurze Anmerkung zur Notation: \( \sigma \) wird in diesem Bericht für die Standardabweichung eines Datensatzes verwendet, während \( \alpha \) der Fehler einer bestimmen Größe ist. Öfters fallen diese beiden zusammen (bzw. \( \alpha \) wird auf \( \sigma \) gesetzt), was für Verwirrung sorgen kann, weshalb dies im Text immer erwähnt wird.
%Als Faustregel kann man sich merken, dass \( \sigma \) die Eigenschaft eines Datensatzes ist und \( \alpha \) sich nur auf einzelne Werte bezieht.