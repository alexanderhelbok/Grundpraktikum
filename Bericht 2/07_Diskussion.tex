% !TeX root = 00_Vorlage.tex
% !TeX spellcheck = de_DE
\section{Diskussion und Schlussfolgerung}
\label{sec:schlussfolgerung}

Die Ergebnisse der Ausmessung des Erdmagnetfeldes werden in \autoref{table:2} den Literaturwerten gegenübergestellt und man erkennt, dass die beiden Werte nicht miteinander vereinbar sind. Das liegt sehr wahrscheinlich daran, dass am Versuchsort eine Vielzahl an metallischen Objekten vorhanden waren, die Magnetfelder beeinflussen oder sogar selbst erzeugen. Der Versuch wurde nämlich in einem Gebäude aus Stahlbeton durchgeführt, in einem Raum voller Elektronik und oberhalb Labore, in welchen Experimente mit elektrischen und magnetischen Feldern durchgeführt werden. Es kommt daher zu systematischen Abweichungen, die verringert werden können, indem man die Messung fern von metallischen Objekten wiederholt.
\begin{center}
	\captionof{table}{Vergleich zwischen experimentell bestimmten und Literaturwerten für die Stärke und Inklination des Erdmagnetfeldes.}
	\begin{tabular}
		{@{\extracolsep{5mm}} r c c}
		\toprule
		\makecell[t]{}
		&   {\makecell[t]{Stärke \( |\vec{B}| \)}}
		&   {\makecell[t]{Winkel \( \theta \)}}\\
		\midrule
		gemessen & \( 63.4(9) \unit{\micro T} \) & \( \ang{53.3(6)} \) \\
		Literatur & \( 48.40(15) \unit{\micro T} \) & \( \ang{63.5(2)} \) \\
		\bottomrule
	\end{tabular}
	\label{table:2}
\end{center}

Der Elektromotor hat sich wie erwartet, zwei Mal im und zwei Mal gegen den Uhrzeigersinn gedreht und bestätigt somit die Theorie. Der Versuch dient aber nur der qualitativen Bestimmung der Polung des Permanentmagneten. Weder die Stärke des Magneten, noch der Strom oder das Vorzeichen der Ladungsträger lässt sich dadurch ermitteln.

Die Messung des von einem Strom erzeugten Magnetfeld bestätigt zwar qualitativ die Theorie (lineares Verhalten), ist aber nicht mit den theoretisch bestimmten Werten vereinbar. Gemessen wurde ein Strom \( I = 8.65(7) \unit{A} \), während die Theorie einen Wert \( I_{\text{calc}} = 5.6(4) \unit{A} \) vorhersagt.

Unser Wert weicht also deutlich vom theoretischen Wert ab, was mehreren Ursachen haben kann. Einerseits entlädt sich die Batterie bei einem Kurzschluss rapide, wodurch sich die Spannung und damit der fließende Strom reduziert. Um den Strom zeitlich konstant zu halten, müsste man eine andere Spannungsquelle verwenden, die direkt am Stromnetz hängt und sich daher nicht entlädt. Zweitens wurde der Versuch in einem Gebäude voller metallischer Gegenstände durchgeführt, welche das vom Leiter erzeugt Magnetfeld verzerren. Um dieses Problem zu beheben, müsste man den Versuch an einem Ort durchführen, der möglichst Metallfrei ist.

Die statistische Unsicherheit ist bei der Systematik zweitrangig, kann aber weiter mit einer kontinuierlichen und vor allem konstanten Stromquelle reduziert werden, dann müsste man den Stromfluss gar nicht mehr unterbrechen, sonder könnte bei fließendem Strom den Abstand des IOLabs erhöhen.
