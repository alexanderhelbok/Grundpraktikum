% Header: Here are all packages used and some additional definitions
%%%%%%%%%%%%%%%%%%%%%%%%%%%%%%%%%%%%%%%%%%%%%%%%%%%%%%%%%%%%%%%%%%%

\documentclass[11pt,a4paper]{scrartcl}
\usepackage[margin=2.5cm]{geometry}
\usepackage[onehalfspacing]{setspace}
\usepackage{graphicx} % zum Einbinden von Graphiken
\usepackage[breaklinks=true,colorlinks=true,linkcolor=blue,urlcolor=blue,citecolor=blue]{hyperref} % f. Referenzen
\usepackage{amsmath,amsthm,amssymb} % Mathematik Umgebung 
\usepackage{icomma} % Intelligentes Komma, das den richtigen Abstand zwischen Dezimalzahlen als auch in Formeln wählt.
\usepackage[ngerman]{babel} % Deutsche Bezeichnungen bei Inhaltsangabe etc
\usepackage[T1]{fontenc}    % andere Schriftsatzkodierung für richtige Silbentrennung bei Umlauten
\usepackage[locale = DE,space-before-unit=true,per-mode = symbol]{siunitx} % Bessere Einheiten
\usepackage{booktabs,multirow} % Pakete zur Erstellung von Tabellen
\usepackage{placeins} % Definiert den Befehl “\FloatBarrier”, der die Ausgabe der davor eingebundenen Bilder erzwingt, befor der Text weiter geht. (Mit vorsicht zu verwenden)
\usepackage[natbib,abbreviate=true,doi=false,style=numeric-comp,giveninits=true,sorting=none]{biblatex} % Modernes Paket zur Erzeugung von Bibliografien (benötigt biber!)
\usepackage{csquotes} % Fortgeschrittene Funktionen für Zitate, für die deutsche Form der Anführungszeichen bei Referenzen
\addbibresource{MyBibliography.bib} % Ort der .bib Datei, die die Datenbank für Literatur/Referenzen enthält.

\graphicspath{{Bilder/}}

\DeclareSIUnit{\dBm}{dBm}
\DeclareSIUnit[per-mode=reciprocal]\WN{\per\centi\meter}

%%%%%%%%%%%%%%%%%%%%%%%%%%%%%%%%%%%%%%%%%%%%%%%%%%%%%%%%%%%%%%%%%%%
\begin{document}
%
\titlehead{\includegraphics[width=5cm]{logo.jpg}}
\title{Titel des Berichts}
\author{Student Eins\thanks{\href{mailto:Email.Student1@uibk.ac.at}{Email.Student1@uibk.ac.at}}, Student Zwei\thanks{\href{mailto:Email.Student2@uibk.ac.at}{Email.Student2@uibk.ac.at}}}
\date{\today}
\maketitle
\vfill
\section*{\abstractname}
\textit{Wenn man nur 2 Minuten hätte um einem Kollegen/einer Kollegin die Arbeit zu erklären, was soll er/sie davon wissen?}
\thispagestyle{empty}
%
%
\tableofcontents
\thispagestyle{empty}
\cleardoublepage
\pagenumbering{arabic} 
\newpage
%
%
\section{Einleitung}
\label{sec:Einleitung}
%
\textit{Worum geht es in dem Versuch? Was ist die Hauptfrage und warum ist sie interessant?}\\
Im Anhang~\ref{sec:Latex} finden Sie einige Informationen, die hilfreich für den Einstieg in \LaTeX{} sein könnten.
%
\section{Grundlagen und Theorie}
\label{sec:Grundlagen}
%
\textit{Was sind die wichtigen physikalischen Grundlagen?}\\
Tabelle \ref{tab:exapltab} zeigt ein typisches Beispiel für eine Tabelle in einer Auswertung.
\begin{table}[htbp]
\centering
\caption{\label{tab:exapltab}Dies hier ist eine typische Tabelle. Die Werte haben, wenn nicht anders angegeben, eine Unsicherheit von \SI{0,4}{\nano\meter}. Da die lateralen Dimensionen von Teilchen D (in einer Dimension) stark vom Mittelwert abweichen, wurde der Mittelwert zusätzlich ohne dieses Teilchen berechnet.}
\begin{tabular}{cccccc}
\toprule
\multicolumn{1}{c}{Teilchen}	& \multicolumn{1}{c}{Länge / \si{\nano\meter}}		& \multicolumn{1}{c}{Breite / \si{\nano\meter}}	& \multicolumn{1}{c}{Höhe / \si{\nano\meter}} \\
\midrule
\multirow{1}{*}{A} & $34,5$	& $25,6$		& $4,6$		  \\
\multirow{1}{*}{B} & $34,4$	& $28,0$		& $4,0$		  \\
\multirow{1}{*}{C} & $31,9$	& $27,4$		& $4,6$		  \\
\multirow{1}{*}{D} & $49,1$	& $34,0$		& $4,6$		 \\
\multirow{1}{*}{E} & $35,2$	& $26,1$		& $4,9$		 \\
\multirow{1}{*}{F} & $27,2$	& $23,6$		& $4,3$		\\  
\multirow{1}{*}{G} & $27,2$	& $23,6$		& $4,3$		  \\
\midrule
\multirow{1}{*}{Mittelwert:} & $35\pm6$	& $28\pm3$		& $4,7\pm0,6$	\\	  
\multirow{1}{*}{Mittelwert} & \multirow{2}{*}{$33\pm3$}	& \multirow{2}{*}{$27\pm2$}		& \multirow{2}{*}{}	\\	  
\multirow{1}{*}{ohne Teil.\ D:} & 	& & \\	  
\bottomrule
\end{tabular}
\end{table}
%
\section{Experiment und Aufbau}
\label{sec:ExpAufb}
%
\textit{Was würde jemand brauchen, um den Versuch nachzuvollziehen und zu wiederholen?}\\
Als Beispiel für eine Gleichung ist hier die Formel für das Verhalten eines verkippbaren Interferenz-Filters angegeben. Dabei verschiebt sich die Wellenlänge $\lambda\left(\Phi\right)$ der Filterkante bei größeren Winkeln $\Phi$, unabhängig von der Kipp"=Richtung, zu kürzeren Wellenlängen~\cite{TiltFilter}:
\begin{equation}
\lambda\left(\Phi\right)=\lambda\left(0\right)\sqrt{1-\frac{\sin^2\Phi}{n_\text{eff}^2}}\;.
\label{eq:TiltWel}
\end{equation}
$n_\text{eff}$ ist der effektive Brechungsindex.
%
\section{Ergebnisse}
\label{sec:Ergebnisse}
%
\textit{Was ist(sind) die gemessene Antwort(en) auf die Hauptfrage(n)?}\\
Abbildung \ref{fig:examplfig} zeigt ein Beispiel für eine Abbildung in Haupttext.
\begin{figure}[htb!]
 \centering
 \includegraphics[width=0.65\textwidth]{good_example_plot}
 \caption{\label{fig:examplfig}Ein typischer Graph in einer Auswertung.
 }
\end{figure}
%
\section{Schlussfolgerungen}
\label{sec:Schlussf}
%
\textit{Was ist die Endantwort und soll ihr vertraut werden?  Wie hätte man den Versuch anders oder besser durchführen können?}\\
Die Referenz~\cite{GP1StromSpannung} soll ein Beispiel sein, wie man ein Praktikums"=Skript zitieren sollte. 
%
\input{Anhang} % Der Anhang ist in einer externen Datei "Anhang.tex" und wird hier in das Dokument eingefügt.

\printbibliography[]
\vfill
\section*{Erklärung}

Hiermit versichern wir, dass der vorliegende Bericht selbständig verfasst wurde und alle notwendigen Quellen und Referenzen angegeben sind.

\begin{tabular}{@{}p{2.5in}p{2.5in}@{}}
 \\[5\bigskipamount]
  \dotfill & \dotfill \\
  Student 1 & Date \\[5\bigskipamount]
  \dotfill & \dotfill \\
 Student 2 & Date \\
  \centering
  
\end{tabular}

\end{document}
