% !TeX spellcheck = de_DE
\documentclass{alex_gp}

\name{Alexander Helbok}
\course{Grundpraktikum}
\hwnumber{Fehlerrechnung}


\begin{document}
\renewcommand{\labelenumi}{\alph{enumi})}


\begin{mybox}{Richtiges Runden}
	\centering \( x = 1.2736 \unit{m};\qquad \alpha_x = 0.25034 \unit{m} \)
	\tcblower
	\begin{enumerate}
		\item \( \left(1.274 \pm 0.250 \right) \unit{m} \qquad \xmark \)  Nur eine signifikante Stelle im Fehler
	\tcbline
		\item \( \left(1.3 \pm 0.3 \right) \qquad \xmark \)  Keine Einheiten!
	\tcbline
		\item \( 1273 \pm 250 \unit{mm} \qquad \xmark \)  Nur eine signifikante Stelle im Fehler
	\tcbline
		\item \( 1.3(3) \unit{m} \qquad \cmark \)
	\tcbline
		\item \( 1.3(25) \unit{m} \qquad \xmark \)  Nur eine signifikante Stelle im Fehler
	\tcbline
		\item \(  (1.3 \pm 0.25034) \unit{m} \qquad \xmark \)  Nur eine signifikante Stelle im Fehler
	\end{enumerate}
\end{mybox}

\begin{mybox}{Fehlerfortpflanzung}
	\centering \( x = (17.4 \pm 0.3) \unit{V};\quad y = (9.3 \pm 0.7) \unit{V} \)
	\tcblower
	\begin{enumerate}
		\item \( z = x - y = 8.1(8) \unit{V} \)
	\tcbline
		\item \( z = 12x + 3y = 237(4) \unit{V} \)
	\tcbline
		\item \( z = 5xy = 8.1(6) \cdot 10^{2} \unit{V^2} \)
	\tcbline
		\item \( z = \tfrac{y^3}{x^2} = 2.7(6) \unit{V} \)
	\tcbline
		\item \( z = x^2 + 3y^2 = 5.6(4) \cdot 10^{2} \unit{V^2}\)
	\tcbline
		\item \( z = \arcsin(\tfrac{y}{x}) = 0.56(5) \)
	\tcbline
		\item \( z = \sqrt{3xy} = 22.0(9) \unit{V} \)
	\tcbline
		\item \( z = \ln(\tfrac{y}{x}) = -0.63(8) \)
	\tcbline
		\item \( z = \tfrac{x}{y^2} + \tfrac{y}{x^2} = 0.23(3) \unit{1/V} \)
	\tcbline
		\item \( z = 2\sqrt{\tfrac{y}{x}} = 1.46(6) \)
	\end{enumerate}
\end{mybox}

\begin{mybox}{Beispiel: Bestimmung der Fallbeschleunigung \( g \)}
	\centering \( x_1 = 5.000(1) \unit{m};\quad x_2 = 17.000(1) \unit{m};\quad t_x = 77283.5(1) \unit{\micro\s} \)
	\tcblower
	\begin{enumerate}
		\item \( v = \tfrac{x_2 - x_1}{t_x} = 155.27(2) \unit{\v} \)
	\tcbline
		\item \(  \)
%		\begin{flalign*}
	%		
%		\end{flalign*}
	\tcbline
		\item \(  \)
%		\begin{flalign*}
		%			
%		\end{flalign*}
	\end{enumerate}
\end{mybox}

\begin{mybox}{Plotten von Daten mit linearem Fit}
	\centering \(  \)
	\tcblower
	\begin{enumerate}
		\item \(  \)
		\tcbline
		\item \(  \)
		\tcbline
		\item \(  \)
		\tcbline
		\item \(  \)
		\tcbline
		\item \(  \)
		\tcbline
		\item \(  \)
		\tcbline
		\item \(  \)
		\tcbline
		\item \(  \)
		\tcbline
		\item \(  \)
		\tcbline
		\item \(  \)
	\end{enumerate}
\end{mybox}

\end{document}