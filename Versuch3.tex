\documentclass{alex_gp}

\name{Alexander Helbok}
\course{Grundpraktikum}
\hwnumber{3}
\spacing{}

\begin{document}
\renewcommand{\labelenumi}{\alph{enumi})}


\begin{mybox}{Eine Feder mit drei Massen}
	\begin{vwcol}[widths={0.6, 0.4}, sep=.8cm, justify=flush,rule=0pt, indent=1em, lines=20] 
		\begin{minipage}[t][0cm][t]{0.6\textwidth}
%			\hspace{0.5cm}
			\begin{tabular}{@{}r rrr @{}}\toprule
				& Versuch1 & Versuch 2 & Versuch 3 \\ \midrule
				\( m \) [kg] & 0.202 & 0.355 & 0.414 \\
				\( \sqrt{1/m}\; [\text{kg}^{-\tfrac{1}{2}}] \) & 2.220 & 1.677 & 1.553 \\
				T [s] & 1.509 & 1.867 & 2.016 \\
				$\omega$ [1/s] & 8.326 & 6.729 & 6.232 \\
				\bottomrule
			\end{tabular}
			\captionof{table}{Gemessene Masse, Schwingungsdauer und Winkelfrequenz der drei Versuche}
			\label{table:1}
		\end{minipage}%
		\newpage
		\lipsum[2]
	\end{vwcol}
	
\end{mybox}

\begin{mybox}{Entwicklung eines Modells für parallele Federn}

\end{mybox}

\begin{mybox}{}

\end{mybox}


\end{document}