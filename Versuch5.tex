% !TeX spellcheck = de_DE
\documentclass{alex_gp}

\name{Alexander Helbok}
\course{Grundpraktikum}
\hwnumber{5}
\spacing{}
\usepackage{biblatex}
\usepackage{stackengine}


\begin{document}


\begin{mybox}{Kalibrierung der Spannungsquellen}
	Ziel dieses Versuches ist es, die Spannung des IOLabs und der drei in Serie geschaltenen Batterien zu bestimmen. Da der verwendete Sensor nur bis \( 3.3 \unit{V} \) geht, verwenden wir einen Spannungteiler, um die Spannung zu halbieren. Der Spannungteiler setzt sich aus zwei in Serie geschaltenen \( 10.0(1) \unit{\ko} \) Widerständen zusammen. Die Spannung wird nach einem Widerstand gemessen, wo sie zur Hälfte abgefallen ist. Das Potential der Quelle ist demnach doppelt so groß wie der gemessene Wert.
	
	Um den Messwert zu bestimmen, wurde über einen Zeitraum von fünf Sekunden der Mittelwert gebildet. Der Fehler der einzelnen Messwerte wurde auf \( 1/2^{11} \) gesetzt, da der Sensor eine Auflösung von 12 Bit hat, wobei der tatsächliche Wert in nur 11 enthalten ist, da ein Bit das Vorzeichen angibt. Die Unsicherheit im Mittelwert wurde auf die Standardabweichung gesetzt, damit (per Definition) zwei Drittel der Daten innerhalb des \( 1\sigma \) Intervalls liegen. Aufgrund der Toleranz der Widerstände, berechnet sich die Spannung der Quelle als
	\begin{equation}\label{eqn:1}
		U_{\text{source}} = \left(\tfrac{R1}{R2} + 1\right) U_{\text{measured}}
	\end{equation}
	Der beste Schätzwert für die Quellenspannung ist das doppelte des gemessenen Wertes, der Fehler ist aber etwas Größer, da der Fehler der Widerstände ihn vergrößert.
	
	Führt man nun die Messung durch erhält man 
	\begin{equation}\label{eqn:2}
		U_{\text{IOLab}} = 3.296(5) \unit{V} \qquad U_{\text{batt}} = 4.520(6) \unit{V} 
	\end{equation}
\end{mybox}

\begin{mybox}{Überprüfung des Ohmschen Gesetzes}
	Um das Ohmsche Gesetz zu überprüfen, wurde ein Widerstand an die Spannungsquelle \( U_{\text{IOLab}} \) angeschlossen. Da wir aus dem vorherigen Versuch die angelegte Spannung kennen und den Widerstandswert ablesen können, lässt sich der fließende Strom über \( I = U/R \) berechnen.
	
	Um dieses Modell zu überprüfen, messen wir den Strom durch den Schaltkreis und vergleichen anschließend den gemessenen Wert mit dem Berechneten. Den Strom messen wir nicht direkt, sondern indirekt über einen sogenannten Shunt-Widerstand, welcher einen sehr geringen Widerstandswert hat (in unserem Fall ist \( R_{\text{shunt}} = 1.00(5) \unit{\ohm} \)) und daher den Schaltkreis nur gering beeinflusst. Wir können den Spannungsabfall am Shunt-Widerstand über einen präzisen Sensor messen, und über das Ohmsche Gesetz den fließenden Strom berechnen. 
	
	Da der High-Gain Sensor nicht geeicht ist und systematisch vom wahren Messwert abweicht, wurden bei jeder Messung die Kabel und somit die Polarität gewechselt. Der Mittelwert der Absolutbeträge dieser beiden Messungen ist die effektive Spannung, die am Shunt-Widerstand angelegt ist. In \autoref{table:1} wurden die gemessenen Spannungen und den daraus errechneten Strom für die Widerstände \( R_1 = 4.70(5) \unit{\ko} \) und \( R_2 = 10.0(1) \unit{\ko} \) eingetragen.
	
	\begin{center}
		\captionof{table}{Gemessene Spannungen \( U^+, U^- \), effektive Spannung \( U_{\text{eff}} \) und daraus errechneter Strom \( I \) für \( R_1 = 4.70(5) \unit{\ko} \) und \( R_2 = 10.0(1) \unit{\ko} \).}
		\begin{tabular}{@{\extracolsep{5mm}} 
				r
				S[table-format=1.4(1)]
				S[table-format=1.4(1)]
				S[table-format=1.4(1)]
				S[table-format=1.2(1)]
			}
			\toprule
			\makecell[t]{\( R \)}
			&   {\makecell[t]{\( U^+ \) in \( \unit{mV} \)}}
			&   {\makecell[t]{\( U^- \) in \( \unit{mV} \)}}
			&   {\makecell[t]{\( U_{\text{eff}} \) in \( \unit{mV} \)}}
			&   {\makecell[t]{\( I \) in \( \unit{mA} \)}}\\
			\midrule
			\( R_1 \) & 0.2159(5) & -0.4230(4) & 0.3195(3) & 0.32(2) \\
			\( R_2 \) & 0.5820(15) & -0.7871(6) & 0.6846(8) & 0.68(3) \\
			\bottomrule
		\end{tabular}
		\label{table:1}
	\end{center}
	Verwendet man nun das Ohmsche Gesetz in der Form \( I = U/R \) und setzt für die Spannung \( U_{\text{IOLab}} = 3.296(5) \unit{V} \) und für den Widerstand jeweils \( R_1 \) und \( R_2 \) ein, erhält man 
	\begin{equation}\label{eqn:Ohm1}
		I_1 = 0.3296(6) \unit{mA} \qquad I_2 = 0.7012(13) \unit{mA}
	\end{equation}
	Der Shunt-Widerstand wurde bei der Berechnung obiger Stromstärken nicht berücksichtigt, da eine Abweichung erst in der siebten Nachkommastelle festzustellen ist, die nicht mehr signifikant ist. 
	Diese in \autoref{table:1} dargestellten Messwerte stimmen mit den Berechneten im Rahmen der Unsicherheit überein, weshalb das Ohmsche Gesetz von diesem Versuch akzeptiert wird.
\end{mybox}

\begin{mybox}{Widerstandsnetzwerk mit einer Spannungsquelle}
	\begin{wrapfigure}[17]{r}{8cm}
		\vspace{-0.5cm}
		\begin{circuitikz}[european]
			\draw (2,0)-- (4,0)
			to[R, l_=$1 \unit{\ohm}$, name=shunt2] (6,0)   -- (6,3)
			to[R, l_=$4.7 \unit{\ko}$,-*, name=R1]	(4,3) -- (4,2)
			to[R, l_=$1 \unit{\ohm}$, -, name=shunt3]	(2,2)
			to[R, l_=$10 \unit{\ko}$, -*, name=R2]	(-2,2);
			
			
			\draw (2,4)	--	(2,5)
			to[rmeter, t=$U_3$]	(4,5) -- (4,4);
			
			\draw (4,0) --	(4,0.9)
			to[rmeter, t=$U_1$]	(6,0.9);
			
			\draw (2,2) --(2,1.1)
			to[rmeter, t=$U_2$] (4,1.1)	--	(4,2);
			
			\draw (4,3)	--	(4,4)
			to[R, l=$1 \unit{\ohm}$, -, name=shunt1]	(2,4)
			to[R, l=$10 \unit{\ko}$, name=R3]	(0,4)
			to[R, l=$4.7 \unit{\ko}$, name=R4]	(-2,4);
			
			\draw (2,0)
			to[battery1, a=\smash{\stackunder[6pt]{\( U_{\text{IOLab}} \)}{$_-$~~~$_+$}}] (-2,0) -- (-2,4);
			
			\node  at (R1.center) {$R_1$};
			\node  at (R2.center) {$R_2$};
			\node  at (R3.center) {$R_3$};
			\node  at (R4.center) {$R_4$};
			\foreach \i in {1, 2, 3}{
				\node  at (shunt\i.center) {shunt};}
		\end{circuitikz}
		\caption{Schaltkreis mit vier Widerständen  \( R_1 = R_4 = 4.70(5) \unit{\ko}, R_2 = R_3 = 10.0(1) \unit{\ko} \). An drei Stellen wird über einen Shunt-Widerstand die Spannung gemessen.}
		\label{fig:2}
	\end{wrapfigure}
	In diesem Versuch werden die Kirchoffschen Regeln auf ihre Gültigkeit überprüft, indem der fließende Strom an manchen Stellen gemessen wird. Dafür wurde ein etwas komplizierteres Netzwerk aus Widerständen aufgebaut, welches in \autoref{fig:2} dargestellt ist. Es wird an drei Stellen der Strom über einen Shunt-Widerstand bestimmt und anschließend mit dem theoretisch bestimmten Wert verglichen. 
	
	Für die theoretische Bestimmung der Ströme werden drei Maschen aufgestellt, wobei die Shunt-Widerstände hier wieder nicht berücksichtigt werden. Zudem teilt sich der Strom nach \( R_1 \), sodass die Summe der Ströme \( I_2 \) und \( I_3 \) durch die Voltmeter \( U2 \) und \( U_3 \) dem Strom \( I_1 \) durch \( U_1 \) entspricht.
	\begin{align}\label{eqn:mesh1}
		\text{\RN{1}} &: U_{\text{IOLab}} - I_1R_1 - I_2R_2 = 0 &&\\
		\text{\RN{2}} &: U_{\text{IOLab}} - I_1R_1 - I_3R_3 - I_3R_4 = 0&&\\
		\text{\RN{3}} &: I_2R_2 - I_3R_3 - I_3R_4 = 0 &&
	\end{align}
	
	Mithilfe der drei Gleichungen erhält man für die Ströme
	\begin{equation}\label{eqn:Is}
		I_1 = \frac{U_{\text{IOLab}}}{R_{\text{ges}}} \qquad I_2 = \frac{U_{\text{IOLab}} - R_1I_1}{R_2} \qquad I_3 = \frac{R_2I_2}{R_3 + R_4}
	\end{equation}
	wobei \( R_{\text{ges}} \) der Gesamtwiderstand des Schaltkreises ist.
	
	Die Ströme \( I_1, I_2 \) und \( I_3 \) wurden empirisch über Shunt-Widerstände bestimmt. Dabei wird der Spannungsabfall an einem möglichst kleinen Widerstand gemessen und mithilfe des Ohmschen Gesetzes in Strom umgerechnet. Dies wurde an drei Stellen gemacht und die Ergebnisse sind in \autoref{table:2} dargestellt. Die Unsicherheit der Werte kommt einerseits aus der Toleranz der Widerstände und andererseits aus der statistischen Schwankung vom Messsensor. Hierbei dominiert mit die Unsicherheit des Widerstandes mit einem Beitrag von über \( 99.5 \% \) des Gesamtfehlers.
	
 	\begin{center}
		\captionof{table}{Gemessene Spannungen \( U^+, U^- \), effektive Spannung \( U_{\text{eff}} \) und daraus errechneter Strom \( I \) an drei verschiedenen Stellen im Schaltkreis.}
		\begin{tabular}{@{\extracolsep{5mm}} 
				r
				S[table-format=1.4(1)]
				S[table-format=1.4(1)]
				S[table-format=1.4(1)]
				S[table-format=1.3(1)]
			}
			\toprule
			\makecell[t]{Sensor}
			&   {\makecell[t]{\( U^+ \) in \( \unit{mV} \)}}
			&   {\makecell[t]{\( U^- \) in \( \unit{mV} \)}}
			&   {\makecell[t]{\( U_{\text{eff}} \) in \( \unit{mV} \)}}
			&   {\makecell[t]{\( I \) in \( \unit{mA} \)}}\\
			\midrule
			\( U_1 \) & 0.2066(5) & -0.4104(10) & 0.3085(6) & 0.31(2) \\
			\( U_2 \) & 0.1562(7) & -0.1887(6) & 0.1724(5) & 0.172(9) \\
			\( U_3 \) & 0.1001(3) & -0.1374(3) & 0.1187(2) & 0.119(6) \\
			\bottomrule
		\end{tabular}
		\label{table:2}
	\end{center}

	Um den Spannungsabfall an den einzelnen Widerständen empirisch zu ermitteln, wurde mit dem A7 Sensor vom IOLab eine Spannungsmessung im Knoten links neben \( R_1 \) durchgeführt. Der Abfall \( U_{R1} \) und \( U_{R2} \) lassen sich damit bestimmen, da \( U_{\text{IOLab}} = U_{R1} + U_{R2} \) und die Messung gerade den Spannungsabfall nach \( R_1 \) also \( U_{R1} \) gemessen hat. Um \( U_{R3} \) und \( U_{R4} \) zu bestimmen, bedarf es an einer weiteren Messung zwischen den Widerständen \( R_3 \) und \( R_4 \). Dies in Kombination mit \( U_{R2} = U_{R3} + U_{R4} \) liefert die restlichen Spannungen. In \autoref{table:3} wurden alle sowohl empirisch, als auch rechnerisch ermittelten Werte eingetragen.

	\begin{center}
		\captionof{table}{Vergleich zwischen den empirisch und rechnerisch ermittelten Werten für diverse Spannungen und Ströme.}
		\begin{tabular}{@{\extracolsep{5mm}} 
				r
				S[table-format=1.3(1)]
				S[table-format=1.4(1)]
				r
			}
			\toprule
			\makecell[t]{Messgröße}
			&   {\makecell[t]{Gemessener Wert}}
			&   {\makecell[t]{Berechneter Wert}}
			&   {\makecell[t]{Einheit}}\\
			\midrule
			\( I_1 \) & 0.31(2) & 0.3094(5) & \( \unit{mA} \) \\
			\( I_2 \) & 0.172(9) & 0.1842(3) & \( \unit{mA} \) \\
			\( I_3 \) & 0.119(6) & 0.1253(2) & \( \unit{mA} \) \\
			\( U_{R1} \) & 1.453(2) & 1.454(2) & \( \unit{V} \) \\
			\( U_{R2} \) & 1.843(5) & 1.842(3) & \( \unit{V} \) \\
			\( U_{R3} \) & 0.588(4) & 0.5888(11) & \( \unit{V} \) \\
			\( U_{R4} \) & 1.254(6) & 1.253(2) & \( \unit{V} \) \\
			\bottomrule
		\end{tabular}
		\label{table:3}
	\end{center}

	Alle gemessen Werte in \autoref{table:3} lassen sich mit den „Idealwerten“ vereinbaren; die Kirchoffschen Regeln werden von diesem Versuch bestätigt.
\end{mybox}

\begin{mybox}{Widerstandsnetzwerk 2}
	In diesem Versuch wird eine Schaltung mit drei Widerständen und zwei Spannungsquellen mithilfe des Ohmschen Gesetzes und der Kirchoffschen Regeln analysiert. Die Spannungen entsprechen den in \autoref{eqn:2} bestimmten Werten. In \autoref{fig:3} ist die verwendete Schaltung dargestellt.
	\begin{figure}[H]
		\centering
		\begin{circuitikz}[european]
			\draw (0,0)
			to[R, l=$1 \unit{\ohm}$, name=shunt1] (2,0) -- (6,0)
			to[R, l=$1 \unit{\ohm}$, name=shunt2]	(8,0)
			to[battery1, l_=$U_\text{batt}$, invert]  (8,4)
			to[R, l_=$10 \unit{\ko}$, name=R3]	(4,4)
			to[R, l_=$4.7 \unit{\ko}$, name=R1]	(0,4);
			
			\draw (0,4)
			to[battery1, l_=$U_\text{iOLab}$]	(0,0);
			
			\draw (4,0)
			to[R, l={{{{\rotatebox[origin=c]{90}{$1 \unit{\ohm}$}}}}}, *-, name=shunt3] (4,2)
			to[R, l={{{{\rotatebox[origin=c]{90}{$4.7 \unit{\ko}$}}}}}, -*, name=R2] (4,4);
			
			\draw (5,0)
			to[rmeter, t=$U_2$] (5,2)	--	(4,2);
			
			\draw (0,0)	--	(0,-1)
			to[rmeter, t=$U_1$] (2,-1)	--	(2,0);
			
			\draw (6,0)	--	(6,-1)
			to[rmeter, t=$U_3$] (8,-1)	--	(8,0);
			
			\draw[->,shift={(2,2)}] (120:.7cm) arc (120:-90:.7cm) node at(0,0){$I_{\text{IOLab}}$};
			\draw[<-,shift={(6,2)}] (120:.7cm) arc (120:-90:.7cm) node at(0,0){$I_{\text{batt}}$};
			
			\node  at (R1.center) {$R_1$};
			\node [rotate=90] at (R2.center) {$R_2$};
			\node  at (R3.center) {$R_3$};
			\node [rotate=90] at (shunt3.center) {shunt};
			\foreach \i in {1, 2}{
				\node  at (shunt\i.center) {shunt};}
		\end{circuitikz}
		\caption{Schaltkreis mit drei Widerständen \( R_1 = R_2 = 4.70(5) \unit{\ko}, R_3 = 10.0(1) \unit{\ko} \) und zwei Spannungsquellen. An drei Stellen wird über einen Shunt-Widerstand die Spannung gemessen. Die Richtung der Ströme wurde durch Pfeile gekennzeichnet.}
		\label{fig:3}
	\end{figure}
	
	Bevor wir die Ströme empirisch ermitteln, wollen wir anhand des Ohmschen Gesetzes und der Kirchoffschen Regeln eine Vorhersage bezüglich Stromstärke und Stromrichtung abgeben.	Dafür analysieren wir unseren Schaltkreis mithilfe der Kirchoffschen Gesetze und stellen folgende Gleichungen für unseren Schaltkreis auf.
	\begin{align}\label{eqn:mesh2}
		\text{\RN{1}} &: U_{\text{IOLab}} - I_1R_1 - I_2R_2 = 0 &&\\
		\text{\RN{2}} &: U_{\text{batt}} - I_2R_2 - I_3R_3 = 0 &&\\
		\text{\RN{3}} &: I_2 - I_1 - I_3 = 0 &&
	\end{align}
	
	Aus diesen drei Gleichungen lassen sich die drei Ströme bestimmen, da alle Spannungen und Widerstände bekannt sind. Für die Ströme gilt dann
	\begin{equation}\label{eqn:Is2}
		I_1 = \frac{U_{\text{IOLab}}(R_2 + R_2) - R_2U_{\text{batt}}}{R_1(R_2 + R_3) + R_2R_3} \qquad I_2 = I_1 + I_3 \qquad I_3 = \frac{U_{\text{batt}} -R_2I_1}{R_2 + R_3}
	\end{equation}
	
	Für die Ströme wurden numerische Werte eingesetzt und in \autoref{table:4} eingetragen. Alle Ströme sind positiv, was auf ihre Flussrichtung schließen lässt. In \autoref{fig:3} wurde die Richtung der Ströme eingetragen. 
	
	Experimentell wurden die Ströme wieder über den Spannungsabfall an Shunt-Widerständen bestimmt. Diese Spannungsmessung wurde an jedem Punkt drei Mal wiederholt und daraus ein gewichteter Mittelwert gebildet, um die statistische Messunsicherheit zu reduzieren. Das Vorzeichen des Stromes, bzw. die Stromrichtung erkennt man am Vorzeichen der Spannung. Fließt der Strom vom \(  \text{G}_+ \) zum \( \text{G}_- \) Anschluss, ist das Vorzeichen positiv. Für die erste Messreihe \( N_1 \) wurde der \( \text{G}_- \) Anschluss im Knoten zwischen den drei Shunt-Widerständen festgehalten und der \(  \text{G}_+ \) Anschluss ist die Runde gegangen, danach wurden die zwei Anschlüsse getauscht. Wenn man die Stromrichtungen in \autoref{fig:3} betrachtet, erwarten wir einen negativen Strom in \( U_1 \) und \( U_3 \) zu messen, während \( U_2 \) einen positiven messen sollte.
	
	\begin{center}
		\captionof{table}{Gemessene Spannungen \( U_{N1}, U_{N2} \), effektive Spannung \( U_{\text{eff}} \) und daraus errechneter Strom \( I \) an drei verschiedenen Stellen im Schaltkreis.}
		\begin{tabular}{@{\extracolsep{5mm}} 
				r
				S[table-format=1.4(1)]
				S[table-format=1.4(1)]
				S[table-format=1.4(1)]
				S[table-format=1.3(1)]
			}
			\toprule
			\makecell[t]{Sensor}
			&   {\makecell[t]{\( U_{N1} \) in \( \unit{mV} \)}}
			&   {\makecell[t]{\( U_{N2} \) in \( \unit{mV} \)}}
			&   {\makecell[t]{\( U_{\text{eff}} \) in \( \unit{mV} \)}}
			&   {\makecell[t]{\( I \) in \( \unit{mA} \)}}\\
			\midrule
			\( U_1 \) & -0.3224(5) & 0.1159(3) & -0.2192(3) & -0.219(11) \\
			\( U_2 \) & 0.3372(3) & -0.5410(4) & 0.4391(3) & 0.44(2) \\
			\( U_3 \) & -0.3285(4) & 0.1187(8) & -0.2236(4) & -0.224(11) \\
			\bottomrule
		\end{tabular}
		\label{table:5}
	\end{center}
	
	In \autoref{table:5} wurden die gemessenen Spannungen und daraus errechneten Ströme in \( U_1, U_2 \) und \( U_3 \) eingetragen. Die zwei Spannungen, aus welchen sich die effektive Spannung errechnet wurden nicht wie in \autoref{table:2} nach Vorzeichen sortiert, um die Stromrichtung zu betonen. Aus dem gleichen Grund wurde das Vorzeichen in der effektiven Spannung und im Strom beibehalten. Die gemessenen Werte stimmen also mit den theoretischen Vorhersagen überein.
	
	\begin{center}
		\captionof{table}{Vergleich zwischen den empirisch und rechnerisch ermittelten Werten für diverse Ströme.}
		\begin{tabular}{@{\extracolsep{5mm}} 
				r
				S[table-format=1.3(1)]
				S[table-format=1.4(1)]
				r
			}
			\toprule
			\makecell[t]{Messgröße}
			&   {\makecell[t]{Gemessener Wert}}
			&   {\makecell[t]{Berechneter Wert}}
			&   {\makecell[t]{Einheit}}\\
			\midrule
			\( I_1 \) & 0.219(11) & 0.2343(7) & \( \unit{mA} \) \\
			\( I_2 \) & 0.44(2) & 0.4669(6) & \( \unit{mA} \) \\
			\( I_3 \) & 0.224(11) & 0.2326(5) & \( \unit{mA} \) \\
			\bottomrule
		\end{tabular}
		\label{table:4}
	\end{center}
	
	Wenn man die Werte in \autoref{table:4} vergleicht, sieht man, dass die von der Theorie vorhergesagten Werte im \( 1\sigma \) Intervall der empirischen Werte liegen und sich daher gut vertragen. Zudem weichen letztere sowohl im Nennwert, als auch im Fehler weniger als \( 10 \% \) von den Optimalwerten ab. 
	
	Das Wiederholen der Spannungsmessungen hat den Fehler zwar verbessert, der Großteil kommt aber von der \( 5 \% \) Toleranz des Shunt-Widerstandes. Man müsste einen Widerstand mit kleiner Toleranz verwenden, um den Fehler weiter zu reduzieren.
\end{mybox}


\end{document}