% !TeX root = 00_Vorlage.tex
% !TeX spellcheck = de_DE
\chapter{Einleitung}
\label{sec:einleitung}

Harmonische Oszillationen sind nicht nur, wie in diesem Versuch, in der Mechanik anzutreffen, sondern erstreckt sich von der Elektrodynamik bis hin zur Quantenphysik über alle Teilbereiche der Physik. Da sich das Modell des harmonischen Oszillators gut eignet, um nicht-mechanische Konzepte zu beschreiben und anzunähern, wie zum Beispiel ein Bindungsenergien von Atomen. Aber auch im Alltag begegnet uns dies oft. Mit so vielen Anwendungen ist die physikalische Beschreibung des harmonischen Oszillator eine der wichtigsten Werkzeuge der Physik. Dadurch ist es lohnenswert, im Grundpraktikum eines dieser Modelle mit Hilfe eines Experimentes zu validieren. Hier prüfen wir das Hookesche Gesetz, beziehungsweise die Bestimmung einer Federkonstante $k$.\\
Dafür werden in \autoref{sec:Grundlagen} die benötigten Grundlagen und die Vorgehensweise bei der Fehlerrechnung aufgezeigt. Daraufhin wird in \autoref{sec:aufbau} der Aufbau und die Vorgehensweise der einzelnen Versuche beschrieben. Insbesondere wird auf die Massenbestimmung, die Berechnung der Federkonstante $k$, das Aufstellen eines Modells und anschließend auf die Überprüfung dieses Modells eingegangen. In \autoref{sec:ergebnisse} werden die Ergebnisse der drei Versuche dargestellt und mit der Theorie verglichen.
