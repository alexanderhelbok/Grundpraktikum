% !TeX root = 00_Vorlage.tex
% !TeX spellcheck = de_DE
\section{Einleitung}
\label{sec:einleitung}

Harmonische Oszillationen sind nicht nur, wie in diesem Versuch, in der Mechanik anzutreffen, sondern erstreckt sich von der Elektrodynamik bis hin zur Quantenmechanik über alle Teilbereiche der Physik. Das aus der analytischen Mechanik abgeleitete Modell des harmonischen Oszillators eignet sich auch gut, um nicht-mechanische Konzepte zu beschreiben und anzunähern, wie zum Beispiel die Bindungsenergien von Atomen. Es wird nicht umsonst gescherzt, dass sich alles zu einem harmonischen Oszillator reduziert. Mit so vielen Anwendungen ist die physikalische Beschreibung des harmonischen Oszillator eine der wichtigsten Werkzeuge der Physik. Dadurch ist es lohnenswert, im Grundpraktikum eines dieser Modelle mit Hilfe eines Experimentes zu validieren. Hier wenden wir das Modell des einfachen harmonischen Oszillators auf das Federpendel an.

Dafür werden in \autoref{sec:theorie} die benötigten physikalischen und statistischen Grundlagen aufgezeigt. Daraufhin wird in \autoref{sec:aufbau} der Aufbau und die Vorgehensweise der einzelnen Versuche beschrieben. Insbesondere wird auf die Massenbestimmung, die Berechnung der Federkonstante $k$, das Aufstellen eines Modells und anschließend auf die Überprüfung dieses Modells eingegangen. In \autoref{sec:ergebnisse} werden die Ergebnisse der drei Versuche dargestellt und mit der Theorie verglichen und zum Schluss werden die gewonnen Erkenntnisse diskutiert.

%Ziel des Versuches 