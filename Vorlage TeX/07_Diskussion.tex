% !TeX root = 00_Vorlage.tex
% !TeX spellcheck = de_DE
\chapter{Diskussion und Schlussfolgerung}
\label{chap:schlussfolgerung}

In diesem Kapitel werden die Beobachtungen und Berechnungen der Versuche diskutiert, insbesondere  die Korrektheit, des von uns, Aufgestellten Modells. Da dies das Kernthema dieses Berichtes ist.\\
Vergleicht man nun  die aus vorigen Kapiteln berechneten Werte für $k_i$, in Tabelle ~\autoref{table:Zusammenfassung} dargestellt, erkennt man, dass die Experimentell bestimmten Werte für $k_i$ mit denen von uns entwickelten Modell:
\begin{align}
	k=\sum_{i=1}^{N} k_i &&\text{bzw. bei identischen Federn} && k=N*k
\end{align} 
übereinstimmen.\\
\begin{table}[htbp]
	\centering
	\caption{\label{table:Zusammenfassung}Die experimentell bestimmten und theoretisch berechneten Werte für $k_i$}
	\begin{tabular}[h]{cccc}
		\hline
		& $k_1$ in $N/m$ & $k_2$ in $N/m$ & $k_3$ in $N/m$\\
		\hline
		Experimenteller Wert & 14.00(17) &  27.5(5)  & 41.2(8)\\
		Theoretischer Wert & & 28.0(2) & 42.0(5)\\
		\hline
	\end{tabular}
\end{table}
Die Korrektheit unseres Modell wird ebenfalls dadurch bestätigt, dass es mit dem Literatur Modell übereinstimmt.
%Vlt noch gründe finden um Fehler zuverringern wie ein Messgerät mit höhere Auflösung?
